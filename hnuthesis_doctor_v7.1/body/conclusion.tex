% !Mode:: "TeX:UTF-8"
\clearemptydoublepage
\addcontentsline{toc}{chapter}{结\quad 论} %添加到目录中
\chapter*{结\quad 论}

得益于先进的电子显微学理论和精密加工技术,现代 TEM 能够在原子尺度观察材料的结构。TEM 中的三维重构技术是探究材料结构的有力手段。本文通过理论研究、编程模拟、实验验证等手段,对 TEM 三维重构技术中存在的一些问题进行了探讨,主要围绕神经网络算法抑制缺失锥假象、纳米尺度景深对 3DET 的影响、基于单张 TEM 原子像的三维重构新方法三个方面开展了方法与理论研究。本文得到的主要结论如下:



1) 开发了一种 ART 型的神经网络 3DET 重构算法以抑制缺失锥假象。该算法利用神经网络的高维度优势进行拟合优化,解决较低维度的 3DET 图像重建问题。该算法可以大幅度地抑制缺失锥假象,在一定频率范围内恢复出样品缺失的信息。该算法不同于其他抑制缺失锥假象的算法,它在重构过程中不借助任何先验知识。算法中使用了“多次平均”的方式达到求平滑解的目的,相比于使用正则化等方法而言,这种方式使算法对噪音更加稳健。算法在一些缺失锥假象严重,或假象与噪音混合的情况下,更能显示其优势。这解决了一些陶瓷材料基体的强度比内部的相或者孔隙强所导致的严重的缺失锥假象的问题。

2) 当 STEM 入射电子束的景深达到纳米尺度,小于样品的厚度时,HAADF-STEM 像仍然可用于倾转系列三维重构。但此时只有样品内部局部区域的原子能被正确地重构。正确重构的区域位置、厚度和重构的质量,和入射电子束斑的会聚半角、加速电压、聚焦位置以及其与倾转轴的相对位置有关。保证电子束斑聚焦在与倾转轴同一样品深度,能够最大程度地获得高质量的重构结果。此外,因重元素原子静电势对电子束的作用而带来的提前聚焦现象会干扰对电子束实际束斑位置的精确控制。要保证足够的三维重构分辨率以分辨原子,不仅要考虑电子束的分辨率与景深,还要考虑样品尺寸和样品的元素的影响。本工作在理论上揭示了原子分辨率的三维重构与一般块体材料的三维重构的差别,对实验具有指导意义。

3) 提出了一种切实可行的通过单张高分辨 TEM 原子像的定量分析重建一般晶体表面的原子分辨率三维重构技术。要在原子分辨率下从单张二维图像重构一般的晶体材料的表面,必须采取全局匹配算法和自收敛验证方案。在该方案中,重构结果应是多次独立重构的平均结果。三维重构的分辨率与置信度,与实验图像的质量密切相关。而实验图像的质量往往最容易受非晶污染的影响。如果实际样品表面覆盖的非晶厚度小于 1 nm,则三维重构的结果较可靠。

\quad

\quad


\noindent \hei{本论文的主要创新点和工作展望}

\setlength{\baselineskip}{30pt}
\noindent \hei{1. 本论文的创新点}

\song\setlength{\baselineskip}{20pt}

\begin{enumerate}
	\item[1)] 
	提出了一种 ART 型的神经网络三维重构算法。该算法不引入任何先验知识来引导重构结果,能够有效抑制缺失锥假象,具有广泛的适用性。不同于一般的正则化方法,该算法通过“多次平均”的方式获得平滑数值解,这种方式不易与实验噪音互相干扰,具有良好的抗噪音能力。在研究中使用了复杂的 SiC 样品的倾转系列像来验证该算法的实际适用性。在于其他方法进行对比后,结果表明只有该方法能够正确重构 SiC 样品的形貌,并大幅度抑制缺失锥效应。这种方法可以广泛地应用于一些陶瓷、复合材料的重构,同时重构基体与第二相(或孔洞)。
	\item[2)]
	通过理论模拟,探究了纳米尺度景深下原子分辨率三维重构的可行性。研究发现,当景深小于样品厚度时,三维重构技术只能正确重构样品中的局部区域。另外,研究还发现实际正确重构的区域相对于电子束名义聚焦位置偏上,即存在提前聚焦现象。重元素原子的静电势更强,更容易使电子束提前汇聚,所以重构重元素样品时提前聚焦现象更明显,正确重构的区域更易提前。电子束斑的实际聚焦位置与倾转轴之间的相对关系,决定图像包含的样品信息的具体位置,从而决定重构的质量。这些现象和一般块体材料的重构不同,对实验具有指导意义。
	\item[3)] 
	提出了一种切实可行的通过单张高分辨透射电子显微镜照片定量分析重建一般晶体表面的原子分辨率三维重构技术。该技术采取全局匹配算法和自收敛验证方案,能够估计和定义三维重构的分辨率,并引入置信度来定量探究非晶对重构结果的影响。在研究中使用了该方法重构了 Si[110] 晶体样品的二维原子分辨率透射电镜照片,并且测得该重构的结果在原子柱的高度(欠焦量)和厚度(原子个数)方面的分辨率都是一个原子间距(0.384 nm),其表面覆盖的非晶层厚度小于 1 nm。
\end{enumerate}


\setlength{\baselineskip}{30pt}
\noindent \hei{2. 工作展望}

\song\setlength{\baselineskip}{20pt}

\begin{enumerate}
	\item[1)] 
	NNART 算法的“多次平均”方案耗费大量计算时间,需要寻找一种更好的得到平滑解方法,既不需要重复运算,又能获得高质量的重构,抑制缺失锥假象。。
	\item[2)]
	尽管 NNART 能够很好地避免实验噪音对重构结果的干扰,但是并不能够降低或去除噪音。所以还需要研究更好的降噪算法。不过降噪的方法并不需要局限于 3DET 算法之内。
	\item[3)] 
	将论文中的方法应用到广泛的材料科学研究中,解决实际的材料科学问题。
\end{enumerate}	



\clearemptydoublepage