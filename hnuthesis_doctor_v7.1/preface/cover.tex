% !Mode:: "TeX:UTF-8"

\chnunumer{10532}
\chnuname{湖南大学}
\cclassnumber{TG1}
\cnumber{B1413Z0350}
\csecret{公开}
\cmajor{电子显微学及其在材料科学中的应用}
\cheading{博士学位论文}      % 设置正文的页眉,以及自己的学位级别
\ctitle{透射电子显微镜中的三维重构新方法}  %封面用论文标题,自己可手动断行
\etitle{New methods for three-dimensional reconstruction in the \\transmission electron microscope}
\caffil{材料科学与工程学院} %学院名称
\csubjecttitle{学科专业}
\csubject{材料科学与工程}   %专业
\cauthortitle{研究生}     % 学位
\cauthor{沈若涵}   %学生姓名
\ename{SHEN~~Ruohan}
\cbe{B.E.~(Hunan University) 2014}
\cms{M.S.~()}
\cdegree{dissertation}
\cclass{Doctor of engineering}
\emajor{Materials Science and Engineering}
\ehnu{Hunan~University}
\esupervisor{CHEN Jianghua}
\csupervisortitle{指导教师}
\elevel{Professor}% 导师职称
\csupervisor{陈江华~~教授} %导师姓名
\cchair{胡望宇}
\ddate{2021 年 3 月 26 日}
\edate{December,~2020}

\untitle{湖~~南~~大~~学}
\declaretitle{学位论文原创性声明}
\declarecontent{
本人郑重声明:所呈交的论文是本人在导师的指导下独立进行研究所取得的研究成果。除了文中特别加以标注引用的内容外,本论文不包含任何其他个人或集体已经发表或撰写的成果作品。对本文的研究做出重要贡献的个人和集体,均已在文中以明确方式标明。本人完全意识到本声明的法律后果由本人承担。
}
\authorizationtitle{学位论文版权使用授权书}
\authorizationcontent{
本学位论文作者完全了解学校有关保留、使用学位论文的规定,同意学校保留并向国家有关部门或机构送交论文的复印件和电子版,允许论文被查阅和借阅。本人授权湖南大学可以将本学位论文的全部或部分内容编入有关数据库进行检索,可以采用影印、缩印或扫描等复制手段保存和汇编本学位论文。
}
\authorizationadd{本学位论文属于}
\authorsigncap{作者签名:}
\supervisorsigncap{导师签名:}
\signdatecap{签字日期:}


%\cdate{\CJKdigits{\the\year} 年\CJKnumber{\the\month} 月 \CJKnumber{\the\day} 日}
% 如需改成二零一二年四月二十五日的格式,可以直接输入,即如下所示
% \cdate{二零一二年四月二十五日}
\cdate{\the\year\ 年\ \the\month\ 月\ \the\day\ 日} % 此日期显示格式为阿拉伯数字 如2012年4月25日
%\cabstract{
%现代球差矫正的透射电子显微镜早已突破原子分辨率,并随着设备和技术的发展,其分辨率正不断接近物理极限。一般地,尽管达到了相当高的分辨率,一张透射电子显微镜照片只能反映材料结构的二维信息。而材料各方面的性能直接或间接地决定于其三维结构,所以掌握材料的三维结构信息是材料研究中相当重要的一环。在透射电子显微镜中,三维电子断层成像技术是应用最广泛的三维重构技术,它通过样品的倾转系列线性投影图像来对原始样品三维结构进行重建。由于需要收集大量的倾转系列透射电子显微镜图像,该技术在实际运用中存在许多理论和实验问题。首先,由于常规的透射电子显微镜无法收集样品倾转至高角度的图像,致使重构结果在傅里叶空间中存在一个锥形的信息缺失的区域,通常称为缺失锥,这使得重构的结果中存在假象。另一方面,随着扫描透射电子显微镜的电子束斑的尺度越来越小,高角环形暗场像将成为样品局部深度内信息的光学层析,偏离样品结构的线性投影。此时,原子分辨率的三维电子断层成像技术在理论上是否可行、其与常规块体样品的三维重构之间的异同,需要被研究和准确地理解。另外,三维重构长时的实验过程、大量的数据需求所带来的问题较难被克服。为了回避这些困难,一些电子显微学家另辟蹊径,通过结合像模拟和相关理论,从单张或少量透射电子显微镜照片来分析材料的三维结构信息。这些方法的实验实现过程更加简单快捷,但是在理论上面对许多挑战,是值得探究和完善的新技术。
%
%着眼于上述科学问题,本论文通过理论探究、编程模拟、实验验证等手段开展研究,主要取得如下的创新性结果:
%
%(1)开发了一种代数重构型神经网络三维电子断层成像算法。该算法凭借神经网络的高复杂度,能够在不引入任何先验知识的情况下,大幅度抑制缺失锥假象,恢复出缺失锥中丢失的频率信息。不同于一般的正则化方法,该算法通过“多次平均”的方式获得平滑数值解,这种方式不易与实验噪音互相干扰,具有良好的抗噪音能力。模拟实验的测试表明,该算法能够运用于重构各种不同形态的样品,当材料内部具有成分梯度时,抑或是材料具有复杂的形貌时,该算法也能很好地抑制缺失锥效应,并正确地恢复出缺失的频率信息。实验结果表明,当材料的形貌复杂,且存在实验噪音时,该算法相比于其他方法具有很大的优势。
%
%(2)通过理论模拟,探究了纳米尺度景深下原子分辨率三维电子断层成像技术的可行性。研究发现,当景深小于样品厚度时,三维重构技术只能正确重构样品中的局部区域。而且,研究还发现实际正确重构的区域相对于电子束名义聚焦位置偏上,即存在提前聚焦现象。正确重构的区域位置、尺寸以及重构的质量,和入射电子束斑的会聚半角、加速电压、聚焦位置以及其与倾转轴的相对位置有关。重元素原子的静电势更强,更容易使电子束提前会聚,所以重构重元素样品时提前聚焦现象更明显,正确重构的区域更易提前。电子束斑的实际聚焦位置与倾转轴之间的相对关系,决定图像包含的样品信息的具体位置,从而决定重构的质量。电子束斑聚焦于倾转轴上时,可以获得最好的重构效果。电子束斑偏离倾转轴时,相当于收集到的信息中存在缺失锥。
%
%(3)提出了一种切实可行的通过对单张透射电子显微镜原子像的定量分析重建一般晶体表面的原子分辨率三维重构技术。该技术采取全局匹配算法和自洽性验证方案。重构的最终结果是多次独立重构的平均结果,具有统计意义,同时能够估计和定义三维重构的分辨率。重构后还通过模拟分析,并引入置信度来定量探究非晶对重构结果的影响。我们使用该方法对 Si[110] 晶体样品的二维原子分辨率透射电镜照片进行了重构,并且测得该重构的结果在原子柱的高度(欠焦量)和厚度(原子个数)方面的分辨率都是一个原子间距(0.384 nm),其表面覆盖的非晶层厚度小于 $\textnormal{1 nm}$。
%
%本论文开发了高精度乃至原子分辨率的三维重构的新方法,并在理论上探究和揭示了当三维重构走向原子分辨率时将面对的问题。这些方法和理论还有更广阔的探究空间,能够为材料科学研究提供更好的技术手段。
%
%}

\cabstract{
三维电子断层成像技术是透射电子显微镜中应用最广泛的三维重构技术,它通过样品的倾转系列线性投影图像来对原始样品三维结构进行重建。由于需要收集大量的倾转系列透射电子显微镜图像,该技术在实际运用中存在许多理论和实验问题。首先,由于常规的透射电子显微镜无法收集样品倾转至高角度的图像,致使重构结果在傅里叶空间中存在一个锥形的信息缺失的区域,通常称为缺失锥,这使得重构的结果中存在严重的假象。另一方面,随着扫描透射电子显微镜的电子束斑的尺寸越来越小,高角环形暗场像将成为样品局部的光学层析,偏离样品结构的线性投影。此时,原子分辨率的三维电子断层成像技术在理论上是否可行、其与常规块体样品的三维重构之间的异同,需要被研究和准确地理解。另外,为了回避实验中存在的困难,一些电子显微学家另辟蹊径,通过结合像模拟和相关理论,从单张或少量透射电子显微镜图像来分析材料的三维结构信息。这些方法的实验实现过程更加简单快捷,但是在理论上面对许多挑战,是值得探究和完善的新技术。

着眼于上述科学问题,本论文通过理论探究、编程模拟、实验验证等手段开展了如下研究,并取得了一些创新性结果:

(1)开发了一种代数重构型神经网络三维电子断层成像算法。该算法以反传播神经网络为优化模型求解线性投影方程组,直接从实验投影数据进行图像重建。不同于一般的正则化或约束的代数重构算法,该算法凭借神经网络的高复杂度来重构低维的图像,能够在不引入任何先验知识的情况下,大幅度抑制缺失锥假象,恢复出缺失锥中丢失的频率信息。并且,该算法还使用了一种“多次平均”的方式,在统计意义上获得平滑数值解,这使其不易与实验噪音互相干扰,具有良好的抗噪音能力。模拟测试表明,该算法能够运用于重构各种不同形态的样品。实验结果表明,当材料的形貌复杂,且存在实验噪音时,该算法相比于其他方法具有很大的优势。

(2)通过理论模拟,探究了纳米尺度景深下原子分辨率三维电子断层成像技术的可行性。研究中使用多片层法模拟原子模型的倾转系列高角环形暗场像,再使用同时迭代重构技术对其进行重构。为了避免通道效应,研究中以无序的原子模型为探究对象,对比研究了束斑的加速电压、会聚半角、欠焦量、样品的尺寸、元素等因素对三维重构的影响。
研究发现,当景深小于样品厚度时,三维重构技术只能正确重构样品中的局部区域。而且,实际正确重构的区域相对于电子束名义聚焦位置偏上,即存在提前聚焦现象。电子束斑的实际聚焦位置与倾转轴之间的相对关系,决定图像包含的样品信息的具体位置,从而决定重构的质量。电子束斑聚焦于倾转轴上时,可以获得最好的重构效果。电子束斑偏离倾转轴时,相当于收集到的信息中存在缺失锥。

(3)提出了一种切实可行的通过对单张透射电子显微镜原子像的定量分析重建一般晶体表面的原子分辨率三维重构技术。本文首先设计了一种通过模拟匹配,全局定量分析透射电子显微镜原子像的算法。通过详细的理论模拟探究,证明了对于一般的透射电子显微镜原子像,必须使用全局匹配,才能准确重构其对应的样品的三维形貌。并且,本文还提出了一种自洽性验证方案,取多次独立重构的平均结果重构为最终结果,具有统计意义,且能够估计和定义三维重构的分辨率。重构后还通过定量模拟分析,并引入置信度来定量探究非晶对重构结果的影响。我们使用该方法对 Si[110] 晶体样品的二维原子分辨率透射电镜图像进行了重构,并且测得该重构的结果在原子柱的高度(欠焦量)和厚度(原子个数)方面的分辨率都是一个原子间距(0.384 nm),其表面覆盖的非晶层厚度小于 $\textnormal{1 nm}$。

本论文开发了高精度的三维重构新方法,并在理论上探究和揭示了当三维重构达到原子分辨率时将面对的问题。这些方法和理论还有更广阔的探究空间,能够为材料科学研究提供更好的技术手段。

}

\ckeywords{电子显微学;~~三维电子断层成像技术;~~缺失锥;~~神经网络;~~景深}

\eabstract{
Electron tomography is the most popular three-dimensional reconstruction technique in the transmission electron microscopy. It is realized by reconstruction from a series of linear projections of the materials at different tilt angles. However, the massive demand of experimental data complicates the actual problem and forces the technique to face some extra experimental and theoretical problems. Among them, missing wedge, caused by incomplete specimen tilt range, is a vital problem that introduces severe artifacts in the tomogram. On the other hand, as the beam probe size of the scanning transmission electron microscope is reduced, the high-angle annular dark field images will become optical sections, deviating from the linear projections. In this condition, new problems such that whether tomography is theoretically feasible and what the difference is from the reconstruction of bulk materials at low magnification, need to be addressed. In addition, some researchers have proposed several new methods to skip over the negative factors occurring in the redundant images collection process. By combining with some theoretical or simulation results, these methods are capable of analysing the materials' three-dimensional structures from several transmission electron micrographs. Nevertheless, these new techniques still face some theoretical problems and are worth exploring and improving.

Focusing on the scientific issues mentioned above, the present dissertation conducts researches through theoretical study, programming simulation, experimental verification and other means, and some innovative results are mainly achieved:

(1) A neural network tomography algorithm is developed. It performs the recontruction directly from the experimental data by solving the projection linear equations by a back-propagation neural network model. Different from the regularized or constrained algebraic reconstruction techniques, this algorithm takes advantages of the high complexity of the neural network to minimize missing wedge artifacts and retrieve the missing frequency information without the need for prior knowledge. In addition, the smooth solution resulting from the current algorithm is less susceptible to noise than the results of the general regularization methods as a special average scheme is used in the algorithm. Simulation tests demonstrate that the algorithm can be used to reconstruct various materials with different morphologies. Experimental results show that the algorithm has great advantages over other approaches when the shape of the material is complex and there is experimental noise.

(2) Through theoretical simulation, the feasibility of atomic-resolution tomography under nano-scale depth of field is explored. In this study, tilt series of high-angle annular dark field images of atomic models are simulated by multislice method and reconstructed by simultaneous iterative reconstruction technique. Disordered atomic models are used in the study to avoid channelling effect. The influences of the acceleration voltage, semi-convergence angle, defocus, sample's size and element on the reconstruction are explored. The study finds that when the depth of field is less than the sample' thickness, tomography can only correctly reconstruct the local area of the sample. In addition, the actual correct reconstruction area is higher than the nominal focus position of the electron beam, that is, there is a pre-focusing phenomenon. On the other hand, the relative positions between the beam probe and the tilt axis determines the specific positions in the sample, from which the information collected in the images originates, thereby determining the quality of the reconstruction. When the electron beam probe is focused on the tilt axis, the best reconstruction quality can be obtained. Otherwise, it is equivalent to including missing wedge in the collected information.

(3) A feasible three-dimensional reconstruction technique for reconstructing the surface of general crystals, through quantitative analysis of a single atomic transmission electron micrograph, is proposed. Firstly, a global matching algorithm based on quantitative simulation analysis is developed. By detailed simulation tests, the necessity of the global matching algorithm is proved, for accurate reconstrction from a general atomic transmission electron micrograph. In addition, the technique adopts a self-validation scheme, taking an average result of multiple independent reconstructions as the final result. The resolution of the three-dimensional reconstruction can be estimated from the multiple results. Acorrding to the result and simulation analysis, a confidence factor is introduced to quantitatively explore the influence of amorphous on the reconstruction result. Applying the proposed algorithm to a two-dimensional experimental image from a Si[110] crystal sample, it is shown that an atomic-resolution of one interatomic distance (= 0.384 nm) in three-dimension for both the
height (defocus) and the thickness (atom numbers) of Si atomic columns can be achieved, provided that the covering amorphous layers were less than 1.0 nm in thickness.


This dissertation proposes two new techniques to realise high precision three-dimensional reconstruction, and reveals some new theorectical problems that will be faced in atomic-resolution electron tomography. These new theories and techniques have a lot of potential for better supporting material research. 


}

%\eabstract{
%	The resolution of the modern transmission electron microscope has already been raised up to atomic scale and is still reaching its physical limit with the development of the device and technology. Generally, a transmission electron micrograph even with atomic resolution can only reflect two-dimensional structure information of the materials, which is not enough for the material research, because most of the materials' properties are directly or indirectly determined by their three-dimensional structures. Electron tomography is the most popular three-dimensional reconstruction technique. It is realized by reconstruction from a series of linear projections of the materials at different tilt angles. However, the massive demand of experimental data complicates the actual problem and forces the technique to face some extra experimental and theoretical problems. Among them, missing wedge, caused by incomplete specimen tilt range, is a vital problem that introduces severe artifacts in the tomogram. On the other hand, as the beam probe of the scanning transmission electron microscope is dinimished, the high-angle annular dark field images will become  optical sections, deviating from the linear projections. In this condition, new problems need to be adressed  that whether tomography is theoretically feasible and what the difference is from the reconstruction of bulk materials at low resolution. In addition, some researchers have proposed several new methods to skip over the negative factors occurring in the redundant images collection process. By combining with some theories or simulation results, these methods are capable of analysing the materials' three-dimensional structures from several transmission electron micrographs. The new techniques still face some theoretical problems and are worth exploring and improving.
%	
%	Focusing on the scientific issues mentioned above, the present thesis conducts research through theoretical study, programming simulation, experimental verification and other means, and the following innovative results are mainly achieved:
%	
%	(1) A neural network tomography algorithm is developed. Due to the complexity of the neural network, the algorithm is capable of minimizing missing wedge artifacts and retrieving the missing frequency information without the need for prior knowledge. The smooth solution resulting from the current algorithm is less susceptible to noise than the results of the general regularization methods as a special average scheme is used in the algorithm. Simulation tests demonstrate that the algorithm can be used to reconstruct various materials with different morphologies. Experimental results show that the algorithm has great advantages over other approaches when the shape of the material is complex and there is experimental noise.
%	
%	(2) Through theoretical simulation, the feasibility of atomic resolution tomography under nano-scale depth of field is explored. The study finds that when the depth of field is less than the sample' thickness, tomography can only correctly reconstruct the local area of the sample. In addition, the study also finds that the actual correct reconstruction area is higher than the nominal focus position of the electron beam, that is, there is a pre-focusing phenomenon. The electrostatic potential of the heavy element atoms is stronger, thus it is easier for the electron beam to converge in advance. Therefore, the pre-focusing phenomenon is more obvious when reconstructing the heavy element sample. On the other hand, the relative positions between the beam probe and the tilt axis determines the specific positions in the sample, from which the information collected in the images originates, thereby determining the quality of the reconstruction. When the electron beam probe is focused on the tilt axis, the best reconstruction quality can be obtained. Otherwise, it is equivalent to including missing wedge in the collected information.
%	
%	(3) A feasible three-dimensional reconstruction technique at atomic resolution for reconstructing the surface of general crystals, through quantitative analysis of a single high-resolution transmission electron micrograph, is proposed. The technology adopts a global matching algorithm and a self-validation scheme. The final result is the average result of multiple independent reconstructions, which is statistically significant. The resolution of the three-dimensional reconstruction can be estimated from the statistical data. Acorrding to the result and simulation analysis, a confidence factor is introduced to quantitatively explore the influence of amorphous on the reconstruction result. Applying the proposed algorithm to a two-dimensional experimental image from a Si[110] crystal sample, it is shown that an atomic-resolution of one interatomic distance (= 0.384 nm) in three-dimension for both the
%	height (defocus) and the thickness (atom numbers) of Si atomic columns can be achieved, provided that the covering amorphous layers were less than 1.0 nm in thickness.
%	
%	
%	This thesis proposes two new techniques to realise high precision three-dimensional reconstruction, and reveals some new theorectical problems that will be faced in atomic-resolution electron tomography. These new theories and techniques have a lot of potential for better supporting material research. 
%	
%	
%}

\ekeywords{Electron Microscopy;~~Electron Tomography;~~Missing Wedge;~~Neural Network;~~Depth of Field}

\makecover

\clearpage
