% !Mode:: "TeX:UTF-8"
\clearemptydoublepage
\addcontentsline{toc}{chapter}{致\quad 谢} %添加到目录中
\chapter*{致\quad 谢}

感谢我的博士生导师陈江华教授在我七年的读博生涯中给予我的谆谆教诲。陈老师曾教导我,对事要知行合一,格物致知,对人要灵活变通,辩证统一。我在他对这些道理的践行中,逐渐地领会和成长。无论外界是如何地风雨飘摇,他都尽全力给我们打造起一座象牙塔,让我们所有人都能在略微富余的条件下从事研究工作。同时,他用严谨的治学态度鞭策每一个人,让我们不断地突破能力上限。陈老师的心里也许憧憬着一个对酒当歌人生几何的侠义江湖。他常常将我们比喻成在少林寺习武的比丘,凡要下山闯荡江湖,必先过其十八铜人试炼。他说人外有人,山外有山,在江湖里,还有许多比他内力更深的“老和尚”,他对我们的严格把关如何不是对我们最大的爱护和期盼。而江湖儿女们所崇尚的义字当先、肝胆相照的精神,也正是我们当代青年所欠缺的。在这个精致利己主义横行的时代,我们要如何守住底线,不随波逐流?我相信,总有一天,在“深山老林”里修炼过的我们能找到答案。总有一天,陈老师与我们的“江湖梦”会梦想成真。

感谢伍翠兰教授。伍老师是我们本科材料物理班的班主任,我通过伍老师了解高分辨电镜中心,成为陈老师的学生。伍老师是一个充满激情和正能量的人,我每天都能听到她用嘹亮的嗓音与急促的语言和学生们讨论问题。她嫉恶如仇,当学生利益遭受损失时,她总是如电光火石般地为学生讨公道。我想大概是她的能量非常充盈,能够把母爱散发到每个学生身上。

感谢师兄明文全老师。明师兄是我们电子显微学小组的大师兄,是我们每个人的榜样。他工作认真负责,一丝不苟,全心全意带领师弟师妹从事研究工作。在我读博的七年中,他是我接触最频繁的人之一。他为人异常地沉稳,出生于 90 年的他,常常开玩笑说自己是 89 后,他总是在辛勤奋斗,积极进取。而身为一个“天真烂漫”的 90 后,在和他相处的七年之中,我确实与他发生过很多次的摩擦和观念上的冲突。但最终,我发现,80 后或 90 后只是一个标签,社会对我们每个人的要求都是一样地严苛。也许他比我更早地懂得了这个道理,所以才能前进地如此之迅速。

感谢师姐赖玉香老师。赖师姐是继明文全师兄之后,又一个值得大家学习的榜样。我一度以为,她拥有异于我们常人的基因,所以她能像铁打的一般通宵工作,能把高浓度的烧酒一口吞下。她对待研究的态度非常的严谨和踏实,同时又非常平易近人,这是我和她关系非常融洽的一个重要原因。她活得很从容、很谦卑,但在面对工作时会立马筑起铜墙铁壁,一丝漏洞都无法通行。我深深地知道,她是心有猛虎,细嗅蔷薇。她用小小的身体,扛起大大的责任,追逐大大的理想。

感谢课题组的每一位老师和同学。特别地,感谢徐先东教授对我的指导与鼓励;感谢赵新奇、杨丽、刘珍老师对课题组的付出;感谢茶丽梅、袁定旺、尹美杰、杨修波、凡头文、刘凌红、谢盼老师对我的指导;感谢张勇师兄对我的多次帮助、指导以及合作;感谢王时豪、朱东晖、陶冠辉、陈敬、刘路、尹炎祺、陈兴岩师兄,冯佳妮、顾媛师姐在我刚进入课题组时对我的帮助和指导;感谢洪悦、余雄伟师兄、牛凤姣、刘力梅师姐对我的帮助;感谢何玉涛与我一起探索三维重构和神经网络;感谢邵秦、李石勇、黄家莹、冯思雨、刘瑶与我共同度过许许多多欢乐的时光;感谢我们小组的其余成员席海辉、何忆、陈志逵、陈桂森,他们孜孜不倦的身影同样给予过我很大的激励。

感谢西安交通大学的米少波教授、马传生老师对我科研上的帮助和合作。

感谢我父母对我的养育之恩,感谢他们的理解、支持、陪伴、期待,对他们我无以为报。同样感谢爷爷奶奶外公外婆对我的抚养。感谢所有亲朋好友。特别地,感谢舅公莫国繁先生、大姨王芳芳女士、姑姑施娟峰女士对我的关爱。

感谢给予我帮助的所有人!

\begin{flushright}沈若涵\\2020 年 12 月于湖南大学\end{flushright}




