% !Mode:: "TeX:UTF-8"
\clearemptydoublepage
\addcontentsline{toc}{chapter}{附录A  发表论文和参加科研情况说明}
\vspace{0cm}
\chapter*{附录 A~~~~发表论文和参加科研情况说明}
\vspace{0cm}
\setlength{\parindent}{0em}
\textbf{(一)发表的学术论文}
\begin{publist}
\item \textbf{Shen R H}, He Y T, Ming W Q, Zhang Y, Xu X D, Chen J H. Electron tomography for sintered ceramic materials by a neural network algebraic reconstruction technique. Journal of Materials Science \verb'&' Technology, Accepted. 
\item \textbf{Shen R H}, Ming W Q, Chen J H, He Y T, Mi S B, Ma C S. Feasible atomic-resolution electron tomography for general crystal surfaces by quantitative reconstruction from a high-resolution image. Ultramicroscopy, 2019, 205: 27-38. 
\item \hei沈若涵\song, 明文全, 何玉涛, 陈志逵, 席海辉, 何忆, 陈江华. 景深对 HAADF-STEM 原子分辨率三维重构的影响. 电子显微学报, 2020, 39(5): 526-535. 
\item Yu X W, Chen J H, Ming W Q, Yang X B, Zhao T T, \textbf{Shen R H}, He Y T, Wu C L. Revisiting the hierachical microstructures of an Al-Zn-Mg alloy fabricated by pre-deformation and aging, Acta Metallurgica Sinica (English Letters), 2020, 33: 1518–1526. 
\item Ma Q, Hu W M, Peng D C, \textbf{Shen R H}, Xia X H, Chen H, Chen Y X, Liu H B. Freestanding core-shell Ni(OH)$_2$@MnO$_2$ structure with enhanced energy density and cyclic performance for asymmetric supercapacitors. Journal of Alloys and Compounds, 2019, 803(30): 866-874. 
\item 明文全, 陈江华, 牛凤姣, \hei沈若涵\song, 何玉涛, 陈志逵. 一种基于改进的多层法和GPU加速的透射电镜图像模拟算法和程序. 电子显微学报, 2018, 37(5): 427-435.
\item Ming W Q, Chen J H, Allen C S, Duan S Y, \textbf{Shen R H}. A quantitative method for measuring small residual beam tilts in high-resolution transmission electron microscopy. Ultramicroscopy, 2018, 184: 18-28.
\item Ming W Q, Chen J H, He Y T, \textbf{Shen R H}, Chen Z K. An improved iterative wave function reconstruction algorithm in high-resolution transmission electron microscopy. Ultramicroscopy, 2018, 195: 111-120.
\item Xie P, Han M, Wu C L, Yin Y Q, Zhu k, \textbf{Shen R H}. A high-performance TRIP steel enhanced by ultrafine grains and hardening precipitates. Materials \& Design, 2017, 127: 1-7.

\end{publist}

%\vspace*{1em}
%\textbf{(二)申请及已获得的专利(无专利时此项不必列出)}
%\begin{publist}
%\item XXX,XXX. XXXXXXXXX:中国,1234567.8[P]. 2012-04-25.
%\end{publist}
\vspace*{0em}
\textbf{(二)参与的科研项目}
\begin{publist}
\item	透射电子显微镜定量化原子成像技术及分析仪器平台系统,国家重大科研仪器研制项目(No.11427806)
\end{publist}
\vfill
\hangafter=0\hangindent=0em\noindent

\setlength{\parindent}{0em}
